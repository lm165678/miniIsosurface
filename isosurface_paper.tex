\documentclass[letterpaper]{article}
\usepackage[sumlimits,]{amsmath}

\begin{document}

\section{Intro}

Isosurface extraction

-used in volume visualization, levelset extraction, medical CT imaging techniques, cfd

-movement towards massively parallel architectures

-algorithms must be kept up to date

-make best use of parallel hardware, many core, gpu, mpi--heterogenous

-want to process very large datasets

\begin{itemize}
    \item what are isosurfaces and where are their applications
    \item existing work w.r.t visualization of isosurfaces
    \item purpose of this
    \item many ways of parallelizing code, not easy
    \item what we could've done but didn't do
\end{itemize}

\section{Marching Cubes}
\begin{itemize}
    \item history / reference
    \item the algorithm
    \item how it can be parallelized (diagram?)
    \item how it can be parallelized among different paradigms
\end{itemize}

\section{Flying Edges}
\begin{itemize}
    \item history / reference
    \item the algorithm
    \item how it can be parallelized (diagram?)
    \item how it can be parallelized among different paradigms
\end{itemize}

\section{Programmability}

\begin{itemize}
    \item which codes are easiest to understand? easiest to write?
    \item Differences among algorithms
    \item Differences among paradigms
\end{itemize}

\section{results}

\begin{itemize}
    \item small, med, large tests
    \item dense, not dense tests
    \item FE vs MC discussion
\end{itemize}

\section{conclusion}
\begin{itemize}
    \item summarize\dots
\end{itemize}

\section{stuff}

Hello, Latex world.

\begin{eqnarray}
    \label{eqn:euler}
    e^{i\pi} + 1 &=& 0
\end{eqnarray}

\begin{eqnarray}
    \label{eqn:simple}
    1 + 1 = 2
\end{eqnarray}



\end{document}


